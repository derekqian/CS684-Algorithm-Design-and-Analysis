\documentclass[11pt,letterpaper,oneside]{article}
\usepackage[top=0.5in,left=1in,right=1in,bottom=1in]{geometry}
%\usepackage[top=0.5in,left=1.8in,right=1.8in,bottom=1in]{geometry}
\usepackage{fancyvrb}
\usepackage{colortbl}
\usepackage{graphicx}
\usepackage{url}
\usepackage{amsmath}

%\usepackage{setspace}
%\doublespacing

\title{Homework 3}
\author{JEFFERSON}

\date{}

\begin{document}
\maketitle


\noindent\textbf{Question:}

\vspace{-1em}
\begin{enumerate}
  \itemsep1pt \parskip0pt \parsep0pt
  \item Consider the dual of the first problem described in the book. Instead of cutting rods for maximum profit, let's try buying rods at minimum cost. Assume that one can buy connectors of cost 2 that join two rods, and that rods are available for purchase at the prices listed in the following table (from the book).

\begin{tabular}{cc}
Length & Cost\\ \hline
1 & 1\\
2 & 5\\
3 & 8\\
4 & 9\\
5 & 10\\
6 & 17\\
7 & 17\\
8 & 20\\
9 & 24\\
10 & 30\\
\end{tabular}

\begin{enumerate} 
\item Write pseudocode using memoization or dynamic programming for efficiently finding the cost of a minimum-cost assembly for a rod of length n.
\item Find the big-O worst-case complexity of your algorithm.
\item  (optional) Implement your algorithm, and show the minimum cost for assembling rods of length 1-30 given the above costs.
\end{enumerate}

  \item The main way Rabin-Karp is used today is as a way of searching for multiple pattern strings in a given target string in parallel. The basic idea is to use a Bloom filter to check whether the current rolling hash value might be one of a set of target hashes. If the Bloom filter returns positive, the hash is compared directly with each hash in the set. If a hash matches, then the pattern string is checked against the target string.
\begin{enumerate} 
\item Give pseudocode for the above algorithm.
\item Assume that for a given search for p pattern strings of length m the Bloom filter has been chosen to give a false-positive probability of 0.01, and the rolling hash function's false-positive rate is 0.02. When searching a target string of length n much larger than m, what is the expected ``overhead''? The overhead is computed as:

\begin{tabular}{c}
``extra'' hash comparisons + ``extra'' character compares\\ \hline
optimal ``\hspace{7em}" + optimal ``\hspace{7em}"\\
\end{tabular}

\end{enumerate}

\end{enumerate}


\noindent\textbf{Answers:}
\begin{enumerate}
\item 
\begin{enumerate}
\item The pseudocode using dynamic programming for efficiently finding the cost of a minimum-cost assembly for a rod of length n is shown below.

\vspace{1em}\hspace{2em} BOTTOM-UP-CUT-ROD($p$, $n$)

\hspace{3em}    let $r[0..n]$ be a new array

\hspace{3em}    $r[0] = 0$

\hspace{3em}    for $j = 1$ to $n$
 
\hspace{4em}        $q = p[j]$

\hspace{4em}        for $i = 1$ to $j-1$
 
\hspace{5em}            $q = min(q, p[i] + r[j-i] + 2)$

\hspace{4em}        $r[j] = q$

\hspace{3em}    return $r[n]$

\null

\item The time complexity of this algorithm is $O(n^2)$.
  
This algorithm has a doubly-nested loop structure. The outer loop executes for $n$ times, the inner loop executes for $0$ to $n-1$ respectively. So the total operation in the inner loop will executs for $0 + 1 + ... + (n-1) = n(n-1)/2$ times. As a result, the complexity of the algorithm is $O(n^2)$ comparisons.

\end{enumerate}

\item 
\begin{enumerate}
\item The pseudocode for the algorithm is shown below. $T$ is the target string of length $n$. $P$ is $p$ pattern strings of length $m$. $S$ is the set contains the hashes of all the patterns.

\vspace{1em}\hspace{2em} MULTI-RABIN-KARP-MATCHER($T$, $P$, $S$, $d$, $q$)

\hspace{3em}    $n = T.length$

\hspace{3em}    $m = P_1.length$

\hspace{3em}    $p = P.size$   \hspace{4em}// $P$ contains $p$ patterns denoted as $P_j$

\hspace{3em}    $h = d^{m-1}\ mod\ q$

\hspace{3em}    for $j = 1$ to $p$
 
\hspace{4em}        $p_j = 0$

\hspace{3em}    $t_0 = 0$

\hspace{3em}    for $i = 1$ to $m$   \hspace{4em}// preprocessing
 
\hspace{4em}        for $j = 1$ to $p$
 
\hspace{5em}            $p_j = (dp_j + P_j[i])\ mod\ p$

\hspace{4em}        $t_0 = (dt_0 + T[i])\ mod\ q$

\hspace{3em}    for $s = 0$ to $n-m$    \hspace{4em}// matching
 
\hspace{4em}        if $t_s$ in $S$    \hspace{4em}// determine by bloom filter

\hspace{5em}            for $j = 1$ to $p$    \hspace{4em}// compare directly with each pattern hash
 
\hspace{6em}                if $t_s$ == $p_j$

\hspace{7em}                    if $P_j[1..m]$ == $T[s+1..s+m]$

\hspace{8em}                        print ''Pattern j occurs with shift s"

\hspace{4em}        if $s < n-m$

\hspace{5em}            $t_{s+1} = (d(t_s - T[s+1]h) + T[s+m+1])\ mod\ p$

\null

\item The expected overhead is computed as following.

I assume the overhead means the time used for spurious hits.

``extra" hash comparisons: $0.001*(n-m+1)*p$

``extra" character compares: $0.001*(n-m+1)*p * 0.02*m$

``optimal" hash comparisons: $(1-0.001)*(n-m+1)$

``optimal" character compares: $0.001*(n-m+1)*p * (1-0.02)$

So the overhead is: $\frac{0.001*(n-m+1)*p\ +\ 0.001*(n-m+1)*p*0.02*m}{(1-0.001)*(n-m+1)\ +\ 0.001*(n-m+1)*p*(1-0.002)}$

\end{enumerate}

\end{enumerate}
\end{document}
