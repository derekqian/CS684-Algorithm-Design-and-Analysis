\documentclass[11pt,letterpaper,oneside]{article}
\usepackage[top=0.5in,left=1in,right=1in,bottom=1in]{geometry}
%\usepackage[top=0.5in,left=1.8in,right=1.8in,bottom=1in]{geometry}
\usepackage{fancyvrb}
\usepackage{colortbl}
\usepackage{graphicx}
\usepackage{url}
\usepackage{amsmath}

%\usepackage{setspace}
%\doublespacing

\title{Homework 5}
\author{JEFFERSON}

\date{}

\begin{document}
\maketitle


\noindent\textbf{Question:}

%\vspace{-1em}
Consider the following pseudocode for translating a graph coloring instance into a SAT instance. The pseudocode outputs clauses of a CNF formula that is SAT if and only if the graph $G$ is colorable in $k$ colors. The atoms $v[i,c]$ of the formula are true if variable $v$ should be colored with color $c$.

\vspace{1em}\hspace{2em} SAT-from-GC(graph G=(V, E), limit k colors):

\hspace{3em}     -- enforce the edge constraints 

\hspace{3em}     $\forall$ (v[i], v[j]) in E 

\hspace{4em}         $\forall$ c in 1..k 

\hspace{5em}             emit clause ($\neg$ v[i,c] $\vee$ $\neg$ v[j,c]) 

\hspace{3em}     -- ensure that every vertex is colored 

\hspace{3em}     $\forall$ i in 1..$|V|$ 

\hspace{4em}         emit clause (v[i,1] $\vee$ v[i,2] $\vee$ .. $\vee$ v[i, k]) 

\hspace{3em}     -- ensure that every vertex is uniquely colored 

\hspace{3em}     $\forall$ i in 1..$|V|$ 

\hspace{4em}         $\forall$ c in 1..k 

\hspace{5em}             $\forall$ $c'$ in 1..k 

\hspace{6em}                 if $c \neq c'$

\hspace{7em}                     emit clause ($\neg$ v[i,c] $\vee$ $\neg v[i,c']$)

\null

\begin{enumerate}
  \itemsep1pt \parskip0pt \parsep0pt
  \item What is the big-O worst-case complexity of this translator? What are the metrics for input size? What operations should be counted?
  \item Give pseudocode for the corresponding translator from a satisfying assignment (assignment of true or false to each atom) to a graph coloring.
\end{enumerate}


\null
\noindent\textbf{Answers:}
\begin{enumerate}
\item There are three ``emit clause" invocations in this algorithm. The first one executes for $k\cdot|E|$ times, the second one executes for $|V|$ times, and the third one executes for $k^2\cdot|V|$ times in worst case. The parameter of the first ``emit clause" contains two literals, the parameter of the second one contains $k$ literals, and the parameter of the third one contains 2 literals. The algorithm need to prepare $2k\cdot|E|+k\cdot|V|+k^2\cdot|V|$ literals. Suppose preparing each literal need constant time, then the big-O worst-case complexity of this algorithm is $O(k\cdot|E|+k^2\cdot|V|)$. The metrics for the input size are the number of vertex $|V|$, the number of edges $|E|$ and the number of colors $k$. We use the preparation of literals as the operation.

\item The pseudocode for translating a satisfying assignment to a graph coloring is shown below. The input of this algorithm is a list of atoms $[v(i,c) | 1\leq i \leq |V|, 1 \leq c \leq k]$ returned by the SAT solver. If the list is empty, it means there is no satisfying assignment. The algorithm returns a list of pairs, each pairs $(i,k)$ means vertex $i$ is colored with $k$. An empty list means no such assignment exists.

\vspace{1em}\hspace{2em} GC-from-SAT(v):

\hspace{3em}     -- go through each assignment

\hspace{3em}     a $\leftarrow$ [ ]

\hspace{3em}     if v is empty

\hspace{4em}         return a

\hspace{3em}     $\forall$ i in 1..$|V|$ 

\hspace{4em}         $\forall$ c in 1..k 

\hspace{5em}             if $v[i,k]$

\hspace{6em}                 insert $(i,k)$ into a

\hspace{3em}     return a

\null

\end{enumerate}
\end{document}
